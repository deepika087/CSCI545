% --------------------------------------------------------------
% This is all preamble stuff that you don't have to worry about.
% Head down to where it says "Start here"
% --------------------------------------------------------------
 
\documentclass[12pt]{article}
 
\usepackage[margin=1in]{geometry} 
\usepackage{amsmath,amsthm,amssymb}
\usepackage[T1]{fontenc}
\usepackage{amsmath}
\usepackage{upgreek}
\usepackage{graphicx}
\graphicspath{ {images/RoboticsAssign2_NEw/} }
 
\newcommand{\N}{\mathbb{N}}
\newcommand{\Z}{\mathbb{Z}}
 
\newenvironment{theorem}[2][Theorem]{\begin{trivlist}
\item[\hskip \labelsep {\bfseries #1}\hskip \labelsep {\bfseries #2.}]}{\end{trivlist}}
\newenvironment{lemma}[2][Lemma]{\begin{trivlist}
\item[\hskip \labelsep {\bfseries #1}\hskip \labelsep {\bfseries #2.}]}{\end{trivlist}}
\newenvironment{exercise}[2][Exercise]{\begin{trivlist}
\item[\hskip \labelsep {\bfseries #1}\hskip \labelsep {\bfseries #2.}]}{\end{trivlist}}
\newenvironment{reflection}[2][Reflection]{\begin{trivlist}
\item[\hskip \labelsep {\bfseries #1}\hskip \labelsep {\bfseries #2.}]}{\end{trivlist}}
\newenvironment{proposition}[2][Proposition]{\begin{trivlist}
\item[\hskip \labelsep {\bfseries #1}\hskip \labelsep {\bfseries #2.}]}{\end{trivlist}}
\newenvironment{corollary}[2][Corollary]{\begin{trivlist}
\item[\hskip \labelsep {\bfseries #1}\hskip \labelsep {\bfseries #2.}]}{\end{trivlist}}
\newenvironment{problem}[2][Problem]{\begin{trivlist}
\item[\hskip \labelsep {\bfseries #1}\hskip \labelsep {\bfseries #2.}]}{\end{trivlist}}
 
\begin{document}

\topmargin=-0.45in
\evensidemargin=0in
\oddsidemargin=0in
\textwidth=6.5in
\textheight=9.0in
\headsep=0.25in 
 
% --------------------------------------------------------------
%                         Start here
% --------------------------------------------------------------
 
%\renewcommand{\qedsymbol}{\filledbox}
 
\title{CSCI 545: Homework 2}%replace X with the appropriate number
\author{Deepika Anand} %replace with your name
\maketitle
 
\begin{problem} 1 (a)
\end{problem}
\begin{Answer}
    \begin{equation}
        p_{cm} = \frac {\sum_{i=0}^{n} m_{i} p_{i}^0}{\sum_{i=0}^{n} m_{i}}  
    \end{equation}
    where $p_{i=0}^0$ is defined as vector from origin to point $i$. In other words, representing Frame $i$ in Frame 0.\\
  
    \vspace{5mm}
 
    p_{1}^0 = 
    \[
        \begin{bmatrix}
            l_{1} cos \theta_{1}       \\
            l_{1} sin \theta_{1}       \\            
            0       
        \end{bmatrix}
    \]
    
    \vspace{5mm}
 
    p_{2}^0 = 
    \[
        \begin{bmatrix}
            l_{1} cos \theta_{1} +  l_{2} cos(\theta_{1} + \theta_{2})    \\
            l_{1} sin \theta_{1}  +  l_{2} sin(\theta_{1} + \theta_{2})       \\            
            0       
        \end{bmatrix}
    \]
    \vspace{5mm}
    p_{3}^0 = 
    \[
        \begin{bmatrix}
            l_{1} cos \theta_{1} +  l_{2} cos(\theta_{1} + \theta_{2} +  l_{3} cos(\theta_{1} + \theta_{2} + \theta_{3})    \\
            l_{1} sin \theta_{1}  +  l_{2} sin(\theta_{1} + \theta_{2} +  l_{3} sin(\theta_{1} + \theta_{2} + \theta_{3}))       \\            
            0       
        \end{bmatrix}
    \]
     \vspace{5mm}
     p_{4}^0 = 
    \[
        \begin{bmatrix}
            l_{1} cos \theta_{1} +  l_{2} cos(\theta_{1} + \theta_{2} +  l_{3} cos(\theta_{1} + \theta_{2} + \theta_{3}) + l_{4} cos(\theta_{1} + \theta_{2} + \theta_{3} + \theta_{4})    \\
            l_{1} sin \theta_{1}  +  l_{2} sin(\theta_{1} + \theta_{2} +  l_{3} sin(\theta_{1} + \theta_{2} + \theta_{3}) + l_{4} sin(\theta_{1} + \theta_{2} + \theta_{3} + \theta_{4})       \\     
            0       
        \end{bmatrix}
    \]\\
     \vspace{5mm}
    Since $m_{i} = 1$  for i = 1 to 4.\\
    Hence, \\
    p_{cm} = (1/4) * \[
        \begin{bmatrix}
            4*l_{1} cos_{1} +  3 *l_{2} cos_{12} +  2 *l_{3} cos_{123} + l_{4} cos_{1234}\\
            4*l_{1} sin_{1}  +  3*l_{2} sin_{12}+  2 * l_{3} sin_{123} + l_{4} sin_{1234}   \\     
            0       
        \end{bmatrix}
    \]
\end{Answer}

\clearpage
\begin{problem} 1 (b)
\end{problem}
\begin{Answer}
Geometric Jacobian for 4 - revolute join is given as J = 
\[
        \begin{bmatrix}
            J_{1} & J_{2} & J_{3} & J_{4}
            
        \end{bmatrix}
    \]
For each join $i$, J_{i} = 
\[
        \begin{bmatrix}
            z_{i-1} * (p_{4}^0 - p_{i-1}^0)\\
            z_{i-1}
        \end{bmatrix}
    \]\\
where $z_{i-1}$ is the axis of rotation and $p_{i}^0$ is defined as vector from origin to point $i$. In other words, representing Frame $i$ in Frame 0.
But since we have been asked to consider orientation only. Hence the effective $J_{i}$ will be
\[
        \begin{bmatrix}
            z_{i-1} * (p_{4}^0 - p_{i-1}^0)\\
        \end{bmatrix}
    \]\\


In this case, geometric jacbian J is given as \\
J = \[
        \begin{bmatrix}
            z_{0} * (p_{4}^0) & z_{1} * (p_{4}^0 - p_{1}^0)  & z_{2} * (p_{4}^0 - p_{2}^0)  & z_{3} * (p_{4}^0 - p_{3}^0)\\
        \end{bmatrix}
    \]\\
where z_{0} =  z_{1} = z_{2} = z_{3} = \[
        \begin{bmatrix}
            0  \\
            0 \\
            1
        \end{bmatrix}
    \]\\
\end{Answer}

\clearpage
\begin{problem} 1 (c)
\end{problem}
\begin{Answer}

    Removing all occurrences of $p_{i}^0$ with respective values.
    p_{1}^0 = \[
        \begin{bmatrix}
            l_{1} c_{1}    \\
            l_{1} s_{1} \\
            
        \end{bmatrix}
    \]\\
    \vspace{5mm}
    p_{2}^0 = \[
        \begin{bmatrix}
            l_{1} c_{1} + l_{2}c_{12}    \\
            l_{1} s_{1} + l_{2}s_{12}  \\
            
        \end{bmatrix}
    \]\\
    \vspace{5mm}
    p_{3}^0 = \[
        \begin{bmatrix}
            l_{1} c_{1} + l_{2}c_{12} + l_{3}c_{123}   \\
            l_{1} s_{1} + l_{2}s_{12} + l_{3}s_{123}  \\
            
        \end{bmatrix}
    \]\\
    \vspace{5mm}
    p_{4}^0 = \[
        \begin{bmatrix}
            l_{1} c_{1} + l_{2}c_{12} + l_{3}c_{123} + l_{4}c_{1234}   \\
            l_{1} s_{1} + l_{2}s_{12} + l_{3}s_{123} + l_{4}s_{1234}  \\
            
        \end{bmatrix}
    \]\\
Also, z_{i-1} * p_{i-1}^0 = \frac{\partial p_{4}^0}{\partial \theta_{i}}\\
J = \[
        \begin{bmatrix}
            -l_{1} s_{1} - l_{2}s_{12} - l_{3}s_{123} - l_{4}s_{1234} 
            & -l_{2}_s{12} - l_{3}s_{123} - l_{4}s_{1234} 
            & - l_{3}s_{123} - l_{4}s_{1234}
            & - l_{4}s_{1234}\\
            l_{1} c_{1} + l_{2}c_{12} + l_{3}c_{123} + l_{4}c_{1234}
            & l_{2}c_{12} + l_{3}c_{123} + l_{4}c_{1234}
            & l_{3}c_{123} + l_{4}c_{1234}
            & l_{4}c_{1234}
        \end{bmatrix}
    \]\\
$s_{1234} = sin(\theta_{1} + \theta_{2} + \theta_{3} + \theta_{4})$\\
$c_{1234} = cos(\theta_{1} + \theta_{2} + \theta_{3} + \theta_{4})$ and so on.
\end{Answer}

\clearpage
\begin{problem} 1 (d)
\end{problem}
\begin{Answer}
$J^i$ = Jacobian for $p_{i}$\\
\vspace{5mm}
J^{1} = \[
        \begin{bmatrix}
            -l_{1} s_{1} - l_{2}s_{12} - l_{3}s_{123} - l_{4}s_{1234} \\
            l_{1} c_{1} + l_{2}c_{12} + l_{3}c_{123} + l_{4}c_{1234}\\
        \end{bmatrix}
    \]\\
\vspace{5mm}
J^{2} = \[
        \begin{bmatrix}
            -l_{2}s_{12} - l_{3}s_{123} - l_{4}s_{1234} \\
            l_{2}c_{12} + l_{3}c_{123} + l_{4}c_{1234}\\
        \end{bmatrix}
    \]\\
\vspace{5mm}    
J^{3} = \[
        \begin{bmatrix}
            - l_{3}s_{123} - l_{4}s_{1234} \\
            l_{3}c_{123} + l_{4}c_{1234}\\
        \end{bmatrix}
    \]\\
\vspace{5mm}
J^{4} = \[
        \begin{bmatrix}
            - l_{4}s_{1234} \\
             l_{4}c_{1234}
        \end{bmatrix}
    \]
\end{Answer}

\clearpage
\begin{problem} 1 (e)
\end{problem}
\begin{Answer}
Jacobian for center of mass \\

Use $p_{cm}$ and differentiate with each $\theta_{i}$    

J_{cm} = \frac {1}{4} *  \[
        \begin{bmatrix}
            -4*l_{1} s_{1} - 3*l_{2}s_{12} - 2*l_{3}s_{123} - l_{4}s_{1234}
& 4*l_{1} c_{1} + 3*l_{2}c_{12} + 2*l_{3}c_{123} + l_{4}c_{1234}\\
- 3*l_{2}s_{12} - 2*l_{3}s_{123} - l_{4}s_{1234}
& 3*l_{2}c_{12} + 2*l_{3}c_{123} + l_{4}c_{1234}\\
- 2*l_{3}s_{123} - l_{4}s_{1234}
& 2*l_{3}c_{123} + l_{4}c_{1234}\\
- l_{4}s_{1234}
& l_{4}c_{1234}
        \end{bmatrix}
    \]^T\\

\textbf{T stands for Transpose of this 4X2 matrix.}\\
That is the effective matrix will be of size 2X4    

\end{Answer}

\clearpage
\begin{problem} 1 (f)
\end{problem}
\begin{Answer}
Center of mass Jacobian\\
\vspace{2mm}
\includegraphics[width=12cm, height=6cm]{PartF}
\end{Answer}

\clearpage
\begin{problem} 1 (g)
\end{problem}
\begin{Answer}
Jacobian transpose for inverse kinematics\\
\includegraphics[width=12cm, height=8cm]{PartG_1}\\
\vspace{2mm}
\includegraphics[width=12cm, height=4cm]{PartG_2}\\
\vspace{2mm}
\includegraphics[width=12cm, height=7cm]{PartG_3}\\
Inverse Transpose method requires tuning of $\alpha$. In this case $\alpha = 1$ and hence the graph is a little distorted. However on increasing $\alpha$ the graphs becomes smoother. 
\end{Answer}

\clearpage
\begin{problem} 1 (h)
\end{problem}
\begin{Answer}
Pseudo-inverse for inverse kinematics\\
\includegraphics[width=12cm, height=8cm]{PartH_1}\\
\vspace{2mm}
\includegraphics[width=12cm, height=4cm]{PartH_2}\\
\vspace{2mm}
\includegraphics[width=12cm, height=7cm]{PartH_3}\\
The pseudo-inverse is a least squares best fit approximate solution. and in this case we can see the graph is smooth. Since this graphs doesn't have multiple local minimums therefore since run was enough otherwise multiple random start points are required to avoid stucking in local minima
\end{Answer}

\clearpage
\begin{problem} 1 (i)
\end{problem}
\begin{Answer}
Pseudo-inverse with Null-space optimization\\
\includegraphics[width=12cm, height=8cm]{PartI_1}\\
\vspace{2mm}
\includegraphics[width=12cm, height=6cm]{PartI_2}\\
\vspace{2mm}
\includegraphics[width=12cm, height=7cm]{PartI_3}\\
Null-space optimization allows us to optimize on null-space however the projection on that space never interferes with our optimization equation.This explicit optimization makes the plot smooth.
\end{Answer}

\clearpage
\begin{problem} 1 (j)
\end{problem}
\begin{Answer}
Derivation \\
In this case different DOFs are weighted. So  we need to minimize
\begin{equation}
    \frac{1}{2} \Delta \theta^T W \Delta\theta
\end{equation}
subjected to $\Deltax = J(\theta) \Delta \theta$\\
Effective $\lambda$ is 
\begin{equation}
    F = \frac{1}{2} \Delta \theta^T W \Delta\theta + \lambda^T * (\Delta x - J(\theta)\Delta \theta)
\end{equation}
\begin{equation}
    \frac{\partial F}{\partial \lambda} = 0
\end{equation}
\begin{equation}
    \Delta x = J(\theta) \Delta \theta
\end{equation}

Equating
\begin{equation}
    \frac{\partial F}{\partial \Delta \theta} = 0
\end{equation}
gives,
\begin{equation}
    W \Delta \theta = J^T \lambda
\end{equation}
From 5 and 7
\begin{equation}
    \lambda = W (J J^T)^{-1} J \Delta\theta
\end{equation}
Using this value of $\lambda$ in 7
\begin{equation}
    \Delta\theta = W^{-1} J^{T} (W^{-1} J J^{T})^{-1}
\end{equation}
Now using this equation in code : Weighted pseudo-inverse\\
\includegraphics[width=12cm, height=8cm]{PartJ_1}\\
\vspace{2mm}
\includegraphics[width=12cm, height=6cm]{PartJ_2}\\
\vspace{2mm}
\includegraphics[width=12cm, height=7cm]{PartJ_3}\\
Since weighted pseudo inverse we have used weighted DOF as a result the graph is smooth. 
\end{Answer}

\clearpage
\begin{problem} 1 (k)
\end{problem}
\begin{Answer}
\includegraphics[width=12cm, height=8cm]{PartK_1}\\
\vspace{2mm}
\includegraphics[width=12cm, height=5cm]{PartK_2}\\
\vspace{2mm}
\includegraphics[width=12cm, height=7cm]{PartK_3}\\
In this case we use the advantage of null space optimization as well as weighted DOF hence the arm movement is smooth. 

\end{Answer}


\begin{problem} 2
\end{problem}
\begin{Answer}
The robot arm - NAO screenshot\\
\includegraphics[width=12cm, height=8cm]{2_1}\\

Code
\begin{verbatim}
static int myTarget = 0;
static int 
run_draw_task(void)
{
  int j, i;
  double sum=0;
  double aux;

  if (tau <= -0.5*0 ) {
      if (myTarget == 0) {
          ctarget[RIGHT_HAND].x[_X_] += 0.05/2;
          ctarget[RIGHT_HAND].x[_Z_] -= 0.017;
          tau = 1;
          myTarget = 1;
          run_draw_task();
      } else if (myTarget == 1) {
          ctarget[RIGHT_HAND].x[_X_] += 0.05/2;
          tau = 1;
          myTarget = 3;
          run_draw_task();
      } else if (myTarget == 3) {
          ctarget[RIGHT_HAND].x[_X_] += 0.03;
          ctarget[RIGHT_HAND].x[_Z_] += 0.017;
          tau = 1;
          myTarget = 4;
          run_draw_task();
      } else if (myTarget == 4) {
          ctarget[RIGHT_HAND].x[_X_] += 0.01;
          ctarget[RIGHT_HAND].x[_Z_] += 0.017;
          tau = 1;
          myTarget = 5;
          run_draw_task();
      } else if (myTarget == 5) {
          ctarget[RIGHT_HAND].x[_X_] += 0.009;
          ctarget[RIGHT_HAND].x[_Z_] += 0.017;
          tau = 1;
          myTarget = 6;
          run_draw_task();
      } else if (myTarget == 6) {
          ctarget[RIGHT_HAND].x[_X_] += 0.01/2;
          ctarget[RIGHT_HAND].x[_Z_] += 0.017;
          tau = 1;
          myTarget = 7;
          run_draw_task();
      } else if (myTarget == 7) {
          ctarget[RIGHT_HAND].x[_X_] += 0.025;
          ctarget[RIGHT_HAND].x[_Z_] += 0.01;
          tau = 1;
          myTarget = 8;
          run_draw_task();
      } else if (myTarget == 8) {
          ctarget[RIGHT_HAND].x[_X_] += 0.025;
          ctarget[RIGHT_HAND].x[_Z_] -= 0.007;
          tau = 1;
          myTarget = 9;
          run_draw_task();
      } else if (myTarget == 9) {
          ctarget[RIGHT_HAND].x[_X_] += 0.005;
          ctarget[RIGHT_HAND].x[_Z_] -= 0.03;
          tau = 1;
          myTarget = 10;
          run_draw_task();
      } else if (myTarget == 10) {
          ctarget[RIGHT_HAND].x[_X_] -= 0.03;
          ctarget[RIGHT_HAND].x[_Z_] -= 0.02;
          tau = 1;
          myTarget = 11;
          run_draw_task();
      } else if (myTarget == 11) {
          ctarget[RIGHT_HAND].x[_X_] -= 0.02;
          tau = 1;
          myTarget = 12;
          run_draw_task();
      } else if (myTarget == 12) {
          ctarget[RIGHT_HAND].x[_X_] -= 0.02;
          ctarget[RIGHT_HAND].x[_Z_] += 0.02;
          tau = 1;
          myTarget = 13;
          run_draw_task();
      } else if (myTarget == 13) {
          ctarget[RIGHT_HAND].x[_X_] -= 0.02;
          ctarget[RIGHT_HAND].x[_Z_] += 0.015;
          tau = 1;
          myTarget = 14;
          run_draw_task();
      }  else if (myTarget == 14) {
          ctarget[RIGHT_HAND].x[_X_] -= 0.02;
          tau = 1;
          myTarget = 15;
          run_draw_task();
      } else if (myTarget == 15) {
          ctarget[RIGHT_HAND].x[_X_] -= 0.018;
          ctarget[RIGHT_HAND].x[_Z_] -= 0.005;
          tau = 1;
          myTarget = 16;
          run_draw_task();
      } else if (myTarget == 16) {
          ctarget[RIGHT_HAND].x[_X_] -= 0.02;
          ctarget[RIGHT_HAND].x[_Z_] -= 0.03;
          tau = 1;
          myTarget = 17;
          run_draw_task();
      }
      else {
          freeze();
      }
    return TRUE; 
  }
\end{verbatim}
\clearpage
Screenshot - Code\\
\includegraphics[width=12cm, height=20cm]{2_2}\\

\clearpage
Phase Graph: X-Z\\
\includegraphics[width=12cm, height=8cm]{2_3}\\

Phase Graph: X-Y\\
\includegraphics[width=12cm, height=8cm]{2_4}\\

\clearpage
CLMCPLOT-Data\\
\includegraphics[width=15cm, height=20cm]{2_5}\\

\end{Answer}

\end{document}

